%% Plan_de_projet.tex
%% Copyright 2023 G. Michaud
%% ctrl+alt+j To find something in the PDF SyncTeX

\documentclass{Thesis}
\usepackage{ProjectPackage}
\title{Dosimétrie optique grâce à l'imagerie de polarisation de la radiation Čerenkov
\\\Large{\textit{Une approche pratique pour les mesures initiales et l'acquisition de compétences avec l'équipement}}}
\shorttitle{Dosimétrie optique : Polarisation Čerenkov}
\author{Gérémy Michaud}{G. Michaud}
\teacher{Luc Beaulieu}
\teacher{Louis Archambault}
\usepackage{pifont}
\usepackage{multicol}
\usepackage{setspace}
\usepackage{biblatex}
\setlength{\headheight}{13.87178pt}
\bibliography{Čerenkov.bib}

\begin{document}
\section*{Introduction}
Le présent plan de mesure fournit des directives succinctes pour reproduire les
résultats des articles publiés par \Citeauthor{cloutierAccurateDoseMeasurements2022} \cite{cloutierAccurateDoseMeasurements2022, cloutierDirectInwaterRadiation2022}
dans les domaines de l'émission Čerenkov et de l'imagerie de polarisation.
L'émission Čerenkov se produit lorsqu'une particule chargée se déplace à une vitesse supérieure à celle de la lumière
dans un milieu diélectrique \cite{cerenkovVisibleRadiationProduced1937},
générant ainsi une lumière polarisée utilisable pour détecter et quantifier les distributions de dose dans des applications médicales telles que la radiothérapie \cite{ashrafDosimetryFLASHRadiotherapy2020}.
En exploitant la lumière Čerenkov émise par le milieu irradié, cette méthode prometteuse permet d'évaluer en temps réel la distribution de dose \cite{jarvisCherenkovVideoImaging2014}.

Les deux articles de \Citeauthor{cloutierAccurateDoseMeasurements2022} mentionnés fournissent des méthodes et des résultats expérimentaux
intéressants pour le traitement de l'émission Čerenkov polarisée induite par des faisceaux de photons de d'électrons.
L'objectif du projet est alors de reproduire ces résultats en utilisant des équipements similaires et une méthodologie appropriée.

En adoptant cette approche méthodologique, nous pourrons explorer les aspects clés de
l'émission Čerenkov et de l'imagerie de polarisation, tout en acquérant une expérience
pratique avec les équipements spécifiques. Cette expérience nous permettra alors de contribuer ultérieurement
à l'amélioration des techniques de mesure de la distribution de dose en radiothérapie.

\section*{Objectifs}
Les objectifs de ce projet sont les suivants:
\setlist{nolistsep}
\begin{enumerate}
    \setlength\itemsep{1mm}
    \item Reproduire les résultats de \Citeauthor{cloutierAccurateDoseMeasurements2022}, afin de valider leur méthodologie et leurs résultats expérimentaux.
    \item Acquérir une expérience pratique avec les équipements spécifiques utilisés pour l'émission Čerenkov et l'imagerie de polarisation, en explorant les aspects clés de ces techniques.
    \item Les résultats de ce projet contribueront à l'amélioration des techniques de mesure de dose en radiothérapie en utilisant l'émission Čerenkov, en servant de référence pour de futurs projets avec le système de polariseur PATQUER.
\end{enumerate}

\section*{Méthodes}
Tout d'abord, nous utiliserons un réservoir d'eau positionné à une distance source-peau (DSP) de 100 cm.
Ce réservoir sera irradié en alternance avec des faisceaux de photons et d'électrons de de différentes énergies: 6MV, 18MV, 6MeV et 18MeV.
Le débit de dose sera de 600 UM/min.
Pour les faisceaux de photons, l'espace de phases sera de taille $5 \times \SI{5}{cm^2}$, tandis que pour les faisceaux d'électrons, il sera de taille $6 \times \SI{6}{cm^2}$.

Nous utiliserons une caméra équipée d'un capteur d'image à transfert de charge (CCD) et d'une lentille.
La caméra sera positionnée à une distance de 50 cm du réservoir afin de capturer le signal d'émission Čerenkov généré.
Les images obtenues représenteront le signal Čerenkov projeté le long de l'axe optique en sommant le signal sur l'épaisseur du réservoir d'eau.
Ces images nous permettront de mesurer les distributions de dose projetées.
Pour chaque ensemble de mesures, nous extrairons les profils projetés à la profondeur de dose maximale ($d_{max}$) ainsi que les pourcentages de dose projetés en profondeurs (PPDD).

Afin de comparer et de valider nos résultats, nous reproduirons également les mesures de dose à l'aide de films de radiothérapie.
Ces films seront insérés dans le fantôme d'eau solide imitant les conditions d'irradiation utilisées pour les mesures basées sur l'émission Čerenkov.
Nous effectuerons des mesures de profils en insérant les films à la profondeur de dose maximale, ainsi que des mesures de profondeur en insérant les films entre deux couches de plaques d'eau solide.
Les doses seront sommées le long de la largeur du faisceau, de manière similaire aux mesures d'imagerie Čerenkov.

Pour mesurer la polarisation de l'émission Čerenkov, nous utiliserons une analyse de polarisation à l'aide d'un polariseur linéaire rotatif.
Les images capturées à quatre angles de transmission ($0^\circ$, $45^\circ$, $90^\circ$, $135^\circ$) nous fourniront des informations précieuses sur l'intensité et l'orientation de la lumière polarisée.
En utilisant la loi de Malus et en supposant que le signal sera partiellement polarisé, nous pourrons extraire la contribution polarisée du signal, l'angle moyen de polarisation et la partie non polarisée du signal.

En parallèle, nous utiliserons les simulations Monte Carlo effectuées par \Citeauthor{cloutierAccurateDoseMeasurements2022} à l'aide du logiciel Geant4 pour extraire les distributions polaires et azimutales de l'émission Čerenkov.
Ces simulations ont été réalisées en reproduisant les conditions expérimentales.

Les distributions polaires et azimutales obtenues à partir des simulations seront utilisées pour corriger les distributions de dose Čerenkov déformées en raison de l'anisotropie intrinsèque de l'émission Čerenkov.
La correction sera appliquée en utilisant des facteurs de correction angulaires $C_{\theta}(x, y)$ et $C_{\phi}(x, y)$ pour chaque position dans le réservoir d'eau.

En résumé, la méthode de mesure comprendra la reproduction des conditions expérimentales décrites dans les articles de référence, l'utilisation de films de radiothérapie pour valider les résultats, l'analyse de polarisation à l'aide d'un polariseur linéaire rotatif, et l'utilisation de simulations Monte Carlo pour corriger les distributions de dose Čerenkov en tenant compte de l'anisotropie intrinsèque.

\section*{Équipement}
\begin{multicols}{2}
\begin{itemize}
    \setlength\itemsep{1mm}
    \item Accélérateur linéaire Varian TrueBeam
    \item Cuve d'eau en acrylique de $15 \times 15 \times \SI{20}{cm^3}$
    \item Caméra CCD refroidie Atik 414EX
    \item Lentille de 12 mm de distance focale
    \item Mince film noir opaque couvrant les parois internes de la cuve
    \item Couvertures opaques noires
    \item Polariseur linéaire rotatif XP42-18 d'Edmund Optics
    \item Films de radiothérapie Gafchromic EBT3
\end{itemize}
\end{multicols}

\newpage
\printbibliography
\end{document}