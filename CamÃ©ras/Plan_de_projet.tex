\documentclass{Thesis}
\usepackage{ProjectPackageA22}
\title{Dosimétrie optique grâce à l'imagerie de polarisation de la radiation Cherenkov
\\\Large{\textit{Une approche pratique pour les mesures initiales et l'acquisition de compétences avec l'équipement}}}
\shorttitle{Dosimétrie optique : Polarisation Cherenkov}
\author{Gérémy Michaud}{G. Michaud}
\teacher{Luc Beaulieu}
\teacher{Louis Archambault}
\usepackage{pifont}
\usepackage{multicol}
\usepackage{setspace}
\addbibresource{Cherenkov.bib}

\begin{document}
\section*{Introduction}
Le présent plan de mesure vise à fournir des directives détaillées pour la pratique
et la reproduction des résultats obtenus dans les articles publiés de \citeauthor{cloutier_accurate_2022} \cite{cloutier_accurate_2022,cloutier_direct_2022} concernant les mesures de dose Cherenkov utilisant un équipement spécifique. Les mesures de dose Cherenkov offrent une méthode prometteuse pour évaluer la distribution de dose lors de la radiothérapie externe. Cette approche tire parti de la lumière Cherenkov émise par un milieu irradié, offrant ainsi une possibilité de visualiser et de quantifier la distribution de dose en temps réel.

Ce plan de mesure fournira des instructions étape par étape pour se familiariser avec l'équipement nécessaire, réaliser les mesures de dose Cherenkov et reproduire les résultats décrits dans l'article. Les principales étapes comprennent la préparation de l'équipement, le calibrage de la caméra, le positionnement précis de la cuve d'eau, l'acquisition des images, l'analyse des images, la correction des mesures Cherenkov, la comparaison avec les prévisions du système de planification de traitement (TPS) et l'analyse des écarts.

Il est important de noter que la réalisation de mesures de dose Cherenkov nécessite une connaissance approfondie de la physique des radiations, de la radiothérapie externe et des principes de l'émission Cherenkov. De plus, la manipulation de l'équipement et la mise en œuvre des mesures doivent être effectuées conformément aux protocoles de sécurité et aux réglementations en vigueur dans votre environnement de travail.

En suivant ce plan de mesure, vous serez en mesure d'acquérir une expérience pratique avec l'équipement spécifique et de reproduire les mesures de dose Cherenkov décrites dans l'article. Ces mesures vous permettront d'évaluer la précision des mesures de dose Cherenkov et de comparer les résultats avec les prévisions du TPS utilisé pour la planification du traitement.
\section*{Objectifs}
Les objectifs de ce projet sont:
\setlist{nolistsep}
\begin{enumerate}
    \setlength\itemsep{1mm}
    \item 
    \item 
    \item 
\end{enumerate}

\section*{Méthodes}

\section*{Équipement}
\begin{multicols}{2}
\begin{itemize}
    \setlength\itemsep{1mm}
    \item 
    \item 
    \item 
    \item 
    \item 
    \item 
\end{itemize}
\end{multicols}

\newpage
\printbibliography
\end{document}