\documentclass{Thesis}
\usepackage{ProjectPackageA22}
\title{Dosimétrie optique grâce à l'imagerie de polarisation de la radiation Cherenkov
\\\Large{\textit{Une approche pratique pour les mesures initiales et l'acquisition de compétences avec l'équipement}}}
\shorttitle{Dosimétrie optique : Polarisation Cherenkov}
\author{Gérémy Michaud}{G. Michaud}
\teacher{Luc Beaulieu}
\teacher{Louis Archambault}
\usepackage{pifont}
\usepackage{multicol}
\usepackage{setspace}
\usepackage{biblatex}
\bibliography{Cherenkov.bib}

\begin{document}
\section*{Introduction}
Le présent plan de mesure fournit des directives succinctes pour reproduire les
résultats des articles publiés par \Citeauthor{cloutierAccurateDoseMeasurements2022} \cite{cloutierAccurateDoseMeasurements2022, cloutierDirectInwaterRadiation2022}
dans les domaines de l'émission Cherenkov et de l'imagerie de polarisation.
L'émission Cherenkov se produit lorsqu'une particule chargée se déplace à une vitesse supérieure à celle de la lumière
dans un milieu diélectrique \cite{cerenkovVisibleRadiationProduced1937},
générant ainsi une lumière polarisée utilisable pour détecter et quantifier les distributions de dose dans des applications médicales telles que la radiothérapie \cite{ashrafDosimetryFLASHRadiotherapy2020}.
En exploitant la lumière Cherenkov émise par le milieu irradié, cette méthode prometteuse permet d'évaluer en temps réel la distribution de dose \cite{jarvisCherenkovVideoImaging2014}.

Les deux articles mentionnés fournissent des méthodes et des résultats expérimentaux
intéressants pour le traitement de l'émission Cherenkov polarisée induite par des faisceaux de photons.
L'objectif du projet est alors de reproduire ces résultats en utilisant des équipements similaires et une méthodologie appropriée.

En adoptant cette approche méthodologique, nous pourrons explorer les aspects clés de
l'émission Cherenkov et de l'imagerie de polarisation, tout en acquérant une expérience
pratique avec les équipements spécifiques. Cette expérience nous permettra ultérieurement de contribuer
à l'amélioration des techniques de mesure de la distribution de dose en radiothérapie.

\section*{Objectifs}
Les objectifs de ce projet sont les suivants:
\setlist{nolistsep}
\begin{enumerate}
    \setlength\itemsep{1mm}
    \item Reproduire les résultats de \Citeauthor{cloutierAccurateDoseMeasurements2022}, afin de valider leur méthodologie et leurs résultats expérimentaux.
    \item Acquérir une expérience pratique avec les équipements spécifiques utilisés pour l'émission Cherenkov et l'imagerie de polarisation, en explorant les aspects clés de ces techniques.
    \item Contribuer à l'amélioration des techniques de mesure de la distribution de dose en radiothérapie en utilisant la polarisation de l'émission Cherenkov, en apportant de nouvelles connaissances et des résultats expérimentaux pertinents.
\end{enumerate}

\section*{Méthodes}

\section*{Équipement}
\begin{multicols}{2}
\begin{itemize}
    \setlength\itemsep{1mm}
    \item Accélérateur linéaire Varian TrueBeam
    \item Cuve d'eau en acrylique de $15 \times 15 \times \SI{20}{cm^3}$
    \item Caméra CCD refroidie Atik 414EX
    \item Mince film noir opaque couvrant les parois internes de la cuve
    \item Couvertures opaques noires
    \item 
\end{itemize}
\end{multicols}

\newpage
\printbibliography
\end{document}