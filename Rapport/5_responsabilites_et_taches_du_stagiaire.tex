\section{Responsabilités et tâches du stagiaire}
\leavevmode
\note{\notation{(20\% de la note finale)} Le temps de cette section est le présent. Le contenu principal du rapport qui décrit toutes les étapes franchies et les moyens mis de l'avant pour solutionner les différentes problématiques, une validation des résultats obtenus et la formulation de recommandations pour le futur. Comment la formation en milieu pratique sera porteuse pour la suite des  études de baccalauréat ou pour la fin de la formation universitaire.

{\bfseries{Il est fortement recommandé d'inclure dans cette section des figures, des tableaux, des schémas, des photos ou autres éléments visuels sous peine de perdre des points (jusqu'à \notation{(10\% de la note finale)}).}} Cependant ces éléments doivent apparaître dans le corps du rapport en autant que cela occupe une place raisonnable. Chaque figure et tableau doit correctement être expliqué de manière à être compris indépendamment du texte. Si une section de texte comporte des calculs, croquis de diagrammes, des plans, spécifications techniques ou autres éléments de rapport nécessitant plus d'une demi-page, alors il est souvent préférable d’inclure ces divers éléments en annexe. En fait, pour le corps du texte, on se limite à citer en référence l’élément en question et à décrire les principales hypothèses ou explications nécessaires à la compréhension de ce qui est présenté, les résultats obtenus et la conclusion à laquelle on arrive, interprétation du concept ou autres.}

\subsection{Rôle et contribution du stagiaire}
\note{\notation{(3\% de la note finale)} Vous devez décrire votre rôle dans l'organisation vous ayant accueilli ainsi que vos contributions. Il s'agit de mettre en contexte votre correcteur sur les tâches réalisées et leurs finalités.}

\subsection{Objectifs, problématique, méthodologie/théorie}
\note{\notation{(4\% de la note finale)} Le temps de cette sous-section est le présent ou le passé composé. Plusieurs points sont à décrire ici :
\begin{enumerate}
    \item Objectifs : Les objectifs à atteindre avant la fin de votre stage ou ceux initialement définis lors de votre embauche
    \item Problématique : A quelles problématiques répondent vos tâches.
    \item Méthodologie : {\notation{Uniquement en stage conventionnel}} Décrire la méthodologie permettant de réaliser vos tâches comme les cheminements, etc. 
    \item Théorie : {\notation{Uniquement en stage de recherche}} Décrire toute la théorie permettant la compréhension de votre stage en recherche. La théorie peut être un état de l'art, des formules mathématiques, des notions, etc.
    \item moyens disponibles : Type de rencontre, logiciels, etc.
    \item Échéancier : même si votre échéancier évolue au cours du temps, il est important de rappeler le temps défini pour chaque tâche et la représenter dans le temps. Vous pouvez inscrire l'échéancier atteint en fin de stage.
\end{enumerate}}

\subsection{Description des tâches et des travaux effectués}
\note{\notation{(6\% de la note finale)} Le temps de cette sous-section est le présent ou le passé composé. Vous devez décrire vos tâches qui ont été réalisées durant votre stage ainsi que les différents travaux effectués. Si vous avez réalisé une multitude de tâches, veuillez en choisir seulement 3 ou 4 (les principales). Vous pouvez toujours dire au début de votre paragraphes que vous en avez réalisé une multitude mais que vous vous concentrez sur ces 3/4 tâches ou travaux. Décrivez de manière approfondie!!}

\subsubsection{Mandat : TITRE DU MANDAT \#1}

\subsubsection{Mandat : TITRE DU MANDAT \#2}
\note{Ajouter des mandats au besoin}

\subsection{Résultats / analyses et discussions}
\note{\notation{(4\% de la note finale)} Décrivez la finalité de chacune de vos tâches et travaux.}

\subsubsection{Mandat : TITRE DU MANDAT \#1}

\subsubsection{Mandat : TITRE DU MANDAT \#2}
\note{Ajouter des mandats au besoin}

\subsection{Comparaison avec les attentes du stagiaire}
\note{\notation{(5\% de la note finale)} 
\newline
 {\notation{En cas de premier stage}} Comparaison avec les attentes du stagiaire avant le début du stage. Exposer vos attentes au début de stage et préciser si elles ont été atteinte. Le cas échéant développez.
 \newline
  {\notation{Si vous avez déjà fait un stage, SUPPRIMER CETTE SECTION}}.
 }

\subsection{Comparaison avec le stage précédent}
\note{\notation{(5\% de la note finale)} 
\newline
 {\notation{En cas de premier stage, SUPPRIMER CETTE SECTION}}
 \newline
  {\notation{Si vous avez déjà fait un stage}} Comparaison avec le ou les stages précédents (mentionner la session et l'entreprise). Exposer les différences entre vos tâches, vos préférences, etc.
 }