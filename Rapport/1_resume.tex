
\addcontentsline{toc}{section}{\numberline{}Résumé}
\textbf{Résumé} {\note{{\notation{(5\% de la note finale)}}

Le résumé est le condensé de l’ensemble du rapport. Il doit permettre au lecteur de comprendre l’essentiel de votre stage sans avoir lu la totalité du rapport. Un résumé doit présenter : le but et la nature du travail, les méthodologies utilisées, les principaux résultats et les principales conclusions. Il énumère en outre l'entreprise d'accueil, la période de stage et les tâches du stagiaire.

Exemple : « Ce rapport présente le travail effectué par {\prenomStagiaire}
    {\nomStagiaire} dans le cadre du stage de formation en entreprise, dans le département (de ...) de la compagnie {\nomCompagnie}, pendant la période (du ...). Le stage a consisté à ... » }}