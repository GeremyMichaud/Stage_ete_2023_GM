\addcontentsline{toc}{section}{\numberline{}Liste de symboles et abréviations}
\textbf{Liste de symboles et abréviations}\\
\note{
\begin{multicols}{2}
\begin{acronym}[VBLAST]  % longest acronym to fix width
    \acro{LS}{least squares}\acused{LS}
    \acro{WARP}{Wireless Open-Access Research Platform (Plate-forme de recherche sans fil en libre accès)}\acused{WARP}
    \acro{VBLAST}{Vertical Bell Laboratories Layered Space-Time ()}\acused{VBLAST}
    
\end{acronym}
\end{multicols}
\acresetall

Lors de la première utilisation, vous devez utiliser \textbackslash ac\{WARP\} comme \ac{WARP} qui permet d'écrire l'abréviations au complet. Puis après la première utilisation, vous pouvez introduire votre abréviation ainsi  \textbackslash acs \{WARP\} pour obtenir \acs{WARP}. Si vous n'énumérez pas une abréviation, elle ne s'affichera pas dans la liste ci-dessus. Par exemple, si je utilise seulement \acs{WARP} et \ac{VBLAST}, LS ne s'affichera pas. Pensez à traduire vos abréviations si elles sont en anglais. Pour le cas de \acs{VBLAST}, il s'agit du nom d'un algorithme de détection, vous pouvez le laisser en anglais.}
