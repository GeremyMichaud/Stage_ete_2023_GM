\section{Développement et renforcement des compétences}
\note{\notation{(20\% de la note finale)} L’objectif du rapport est de discuter des acquis professionnels et personnels dans le cadre du stage, au travers des différents mandats. Expliquez comment ce stage participe à votre formation et a été une expérience pertinente pour votre future carrière. Cette partie vous permet d’avoir une réflexion sur la formation pratique et théorique reçue.}
\subsection{Techniques}
\note{\notation{(3\% de la note finale)} En fonction de votre stage, développer sur les techniques de Terrain/laboratoire/programmation, etc. abordées durant votre stage}


\subsection{En ingénierie ou scientifiques}
\note{\notation{(5\% de la note finale)} En fonction de votre stage, discuter des points suivants : Recherche, conception, développement, analyse, suivi, gestion de projet, etc.}

\subsection{Communication}
\note{\notation{(4\% de la note finale)} En fonction de votre stage, discuter du développement et du renforcement des vos compétences en communications en introduisant les points suivants : Rapports, présentation orale, interactions avec différents types d'intervenants, etc.}

\subsection{Réflexion sur la formation pratique et théorique reçue}
\note{\notation{(4\% de la note finale)} Faites une réflexion sur la formatique théorique et pratique.
\begin{enumerate}
    \item Pratique : description d’une méthode de travail acquise ou adoptée (exemple : en développement, en analyse, réunions, etc.) durant le stage ; Organisation du travail adoptée durant le stage : gestion du temps, respect des échéances, gestion des priorités, planification des tâches. Professionnalisme : éthique, contrôle de qualité, santé et sécurité, protection de l’environnement, etc.) Bilan sur l’atteinte des objectifs individuels et de l’employeur fixés
en mi-stage. Commentaires sur la recherche de stage. La formation pré stage. Les ajustements personnels et académiques possibles. Comment votre stage vous permet de mieux cibler le type de carrière que vous envisagez.
   \item Théorique : une question à poser à un enseignant après le stage dans un cours suivi ou à suivre. Un cours à option que le stagiaire choisirait après le stage. Une recommandation que le stagiaire ferait au directeur de son programme.
\end{enumerate}}

\subsection{Bilan des acquis}
\note{\notation{(4\% de la note finale)} Bilan des acquis : En 3-4 lignes, bilan des acquis (par rapport vos attentes avant le stage, les points forts et faibles, vos apprentissages en lien avec votre formation ou vos méthodes de travail, etc.). Avenir professionnel et/ou académique du stagiaire.}