%% AbstractStructure.tex
%% Copyright 2023 G. Michaud
%
% This work may be distributed and/or modified under the
% conditions of the LaTeX Project Public License, either version 1.3
% of this license or (at your option) any later version.
% The latest version of this license is in
%   http://www.latex-project.org/lppl.txt
% and version 1.3 or later is part of all distributions of LaTeX
% version 2005/12/01 or later.
%
% This work has the LPPL maintenance status `maintained'.
% 
% The Current Maintainer of this work is G. Michaud.
%
% This work consists of the file AbstractStructure.tex.


% ---------------
% Input custom pour les résumés scientifiques.
% Par Gérémy "Ken" Michaud
% ---------------

\usepackage[T1]{fontenc}
%\usepackage[french]{babel}
\usepackage[utf8]{inputenc}
\usepackage{csquotes}
\usepackage[hyperref=true,sorting=none,style=phys]{biblatex}
\usepackage[normalem]{ulem}
\usepackage{graphicx}
\usepackage[margin=2cm]{geometry} 
\usepackage{hyperref}
\usepackage{xurl}
\usepackage{booktabs,multirow,multicol,hhline,array}
\usepackage{enumitem}
\usepackage[margin=3cm]{caption}
\usepackage[usenames,dvipsnames]{xcolor}

% Modules mathématiques et scientifiques
\usepackage[version=4]{mhchem}
\usepackage{amsmath,amsthm,amssymb}
\usepackage[locale=FR]{siunitx}
\sisetup{locale = FR,}
\sisetup{detect-weight=true,detect-family=true,separate-uncertainty=true}
\usepackage{wasysym,gensymb,textcomp}
\usepackage{pdflscape} 
\usepackage[section]{placeins}


%  Page d'abstract français
\newcommand*{\resumefrancais}{
\begingroup
    \noindent\makebox[\linewidth]{\rule{\textwidth}{0.4pt}}
	\begin{center}
	{\large\bfseries \titrefrancais}
        \noindent\makebox[\linewidth]{\rule{\textwidth}{0.4pt}}
	\par\vspace{7.5pt}
	\textbf{\premierauteur, \deuxiemeauteur, \troisiemeauteur, \quatriemeauteur, \cinquiemeauteur ~et \sixiemeauteur}\par
	\end{center}
	\vspace{3pt}
	$^1$Département de physique, génie physique et d'optique et Centre de recherche sur le cancer, Université Laval, Québec, Canada

	\vspace{3pt}
	\noindent$^2$Service de physique médicale et de radioprotection, Centre intégré de cancérologie, CHU de Québec-Université Laval et Centre de recherche du CHU de Québec, Québec, Québec, Canada
	
	\vspace{3pt}
	\noindent$^3$Département de physique, génie physique et d'optique et Centre d'optique, photonique et lasers, Université Laval, Québec, Canada
	
	\vspace{3pt}
	\noindent$^4$Patqer Photonique Inc., Québec, Canada\\
    \resume
	\vspace*{\fill}
	\vfill\null
	\noindent\makebox[\linewidth]{\rule{\textwidth}{0.4pt}}
\endgroup}

%  Page d'abstract anglais
\newcommand*{\resumeanglais}{
\begingroup
    \noindent\makebox[\linewidth]{\rule{\textwidth}{0.4pt}}
	\begin{center}
	{\large\bfseries \titreanglais}
        \noindent\makebox[\linewidth]{\rule{\textwidth}{0.4pt}}
	\par\vspace{7.5pt}
	\textbf{\premierauteur, \deuxiemeauteur, \troisiemeauteur, \quatriemeauteur, \cinquiemeauteur ~et \sixiemeauteur}\par
	\end{center}
	\vspace{3pt}
	$^1$Département de physique, génie physique et d'optique et Centre de recherche sur le cancer, Université Laval, Québec, Canada

	\vspace{3pt}
	\noindent$^2$Service de physique médicale et de radioprotection, Centre intégré de cancérologie, CHU de Québec-Université Laval et Centre de recherche du CHU de Québec, Québec, Québec, Canada
	
	\vspace{3pt}
	\noindent$^3$Département de physique, génie physique et d'optique et Centre d'optique, photonique et lasers, Université Laval, Québec, Canada
	
	\vspace{3pt}
	\noindent$^4$Patqer Photonique Inc., Québec, Canada\\
    \abstract
	\vspace*{\fill}
	\vfill\null
	\noindent\makebox[\linewidth]{\rule{\textwidth}{0.4pt}}
\endgroup}


% ----------------------------------------------------
%  Commandes définies localement
% ----------------------------------------------------

\newcommand{\premierauteur}[1]{\renewcommand{\premierauteur}{#1}}
\newcommand{\deuxiemeauteur}[1]{\renewcommand{\deuxiemeauteur}{#1}}
\newcommand{\troisiemeauteur}[1]{\renewcommand{\troisiemeauteur}{#1}}
\newcommand{\quatriemeauteur}[1]{\renewcommand{\quatriemeauteur}{#1}}
\newcommand{\cinquiemeauteur}[1]{\renewcommand{\cinquiemeauteur}{#1}}
\newcommand{\sixiemeauteur}[1]{\renewcommand{\sixiemeauteur}{#1}}

\newcommand{\titrefrancais}[1]{\renewcommand{\titrefrancais}{#1}}
\newcommand{\titreanglais}[1]{\renewcommand{\titreanglais}{#1}}
\newcommand{\resume}[1]{\renewcommand{\resume}{#1}}
\renewcommand{\abstract}[1]{\renewcommand{\abstract}{#1}}